\documentclass{report}
\usepackage[T1]{fontenc} % Fontes T1
\usepackage[utf8]{inputenc} % Input UTF8
\usepackage[backend=biber, style=ieee]{biblatex} % para usar bibliografia
\usepackage{csquotes}
\usepackage[portuguese]{babel} %Usar língua portuguesa
\usepackage{blindtext} % Gerar texto automaticamente
\usepackage[printonlyused]{acronym}
\usepackage{hyperref} % para autoref
\usepackage{graphicx}
\usepackage{indentfirst}
\usepackage{amsmath}

\bibliography{bibliografia}


\begin{document}
%%
% Definições
%
\def\titulo{Computação Quântica}
\def\data{19/11/22}
\def\autores{Gabriel Boia, João Pinto}
\def\autorescontactos{(113167) gabriel.boia@ua.pt, (113093) joaop1@ua.pt}
\def\versao{VERSAO}
\def\departamento{Dept. de Eletrónica, Telecomunicações e Informática}
\def\empresa{Universidade de Aveiro}
\def\logotipo{ua.pdf}
%
%%%%%% CAPA %%%%%%
%
\begin{titlepage}

\begin{center}
%
\vspace*{50mm}
%
{\Huge \titulo}\\ 
%
\vspace{10mm}
%
{\Large \empresa}\\
%
\vspace{10mm}
%
{\LARGE \autores}\\ 
%
\vspace{30mm}
%
\begin{figure}[h]
\center
\includegraphics{\logotipo}
\end{figure}
%
\vspace{30mm}
\end{center}
%
\begin{flushright}
\versao
\end{flushright}
\end{titlepage}

%%  Página de Título %%
\title{%
{\Huge\textbf{\titulo}}\\
{\Large \departamento\\ \empresa}
}
%
\author{%
    \autores \\
    \autorescontactos
}
%
\date{\today}
%
\maketitle

\pagenumbering{roman}

%%%%%% RESUMO %%%%%%
\begin{abstract}
Nesta secção será feito um breve resumo dos conteúdos abordados neste documento. No \autoref{chap.principios} são abordadas várias terioas relacionadas á área da física quântica que explicam o que está por detrás destes computadores quânticos é abordado o princípio da superposição e como se este se relaciona com estes computadores, os princípios base da mecânica quântica que explicam os fundamentos da física quântica e como esta difere da física clássica e o que é o princípio do entanglement quântico e como este se relaciona com a computação quântica.
No \autoref{chap.aplicacoes} são abordadas várias aplicaçôes desta nova tecnologia, primeiro é dado um exemplo de como foi provada a superioridade deste novo tipo de computadores, a seguir são dados exemplos de como aplicar esta tecnologia para melhorar tecnologias já existentes como por exemplo melhorar a encriptação e tornar mais eficientes sistemas informáticos relacionados com a área da saúde e por fim são exemplificadas maneiras em como a computação quântica influência a área farmacêutica e torna mais eficiente o processo de investigação de medicamentos. No \autoref{chap.tipos} são abordados vários tipos de computadores quânticos de modo a dar exemplos das diferentes aproximações e solucões ao problema da computação quântica, neste capítulo são abordados computadores quânticos analógicos e digitais, computadores quânticos universais e computadores quânticos adiabáticos.
\end{abstract}

%%%%%% Agradecimentos %%%%%%
% Segundo glisc deveria aparecer após conclusão...
\renewcommand{\abstractname}{Agradecimentos}
\begin{abstract}
Eventuais agradecimentos.
Comentar bloco caso não existam agradecimentos a fazer.
\end{abstract}


\tableofcontents
% \listoftables     % descomentar se necessário
% \listoffigures    % descomentar se necessário


%%%%%%%%%%%%%%%%%%%%%%%%%%%%%%%
\clearpage
\pagenumbering{arabic}

%%%%%%%%%%%%%%%%%%%%%%%%%%%%%%%%
\chapter{Introdução}
\label{chap.introducao}
Os descobrimentos na área da física quântica, possibilitaram e vão possibilitar diversos avanços tecnológicos e a busca por ainda mais conhecimento levou os seres humanos a desenvolverem computadores com ainda mais capacidades na esperança de aumentar as capacidades de simulação e o conhecimento sobre o universo que detemos.
 \paragraph{}A tecnologia é recente tendo apenas sido proposta nos anos 80 e por isso ainda há muita possibilidade de crescimento nesta área. Já foi provado que estes computadores quânticos conseguem realizar qualquer tarefa realizada pelos computadores clássicos. Algoritmos que levariam anos a ser processados por computadores clássicos podem agora ser processados por computadores quânticos, em um intervalo de tempo bastante menor, o que possibilita que algoritmos que antes não apresentavam solução possam agora ser resolvidos. Ao contrário de computadores clássicos que funcionam á base de bits, que podem tomar o valor de 0 ou 1, computadores quânticos usam quantum bit, mais conhecidos como qubits, que usando os princípios da física quântica podem tomar os valores 0 e 1 em simultâneo.
\paragraph{}Este documento está dividido em quatro capítulos.
Depois desta introdução,
no \autoref{chap.principios} são apresentados principios de física que servem de base á computação quântica,
no \autoref{chap.aplicacoes} são abordadas várias aplicações da computação quântica,
no \autoref{chap.diferencas} são apresentadas algumas diferenças entre computadores quânticos e computadores clássicos,
no \autoref{chap.tipos} são abordados três típos de computadores quânticos.


\chapter{Principios de Física}
\label{chap.principios}
Principio da superposição, interferencia, mecânica quântica e entanglement quântico são principios de física apresentados neste capítulo. Cada princício terá
um subcapítulo próprio.
\paragraph{}Os computadores quânticos operam de acordo com estes principios da física quântica.
\section{Princípio da superposição}
O que permite a um qubit assumir os valores de 0 e de 1 ao mesmo tempo é o principio da superposição que por definição diz que para qualquer sistema linear, a soma das respostas que seriam causadas por cada estímulo individual é igual á resposta da soma dos estímulos causados.De forma que se um input A produzisse uma resposta X e um input B produzisse uma resposta Y então o input (A + B) produz a resposta (X + Y).
Uma função F(x) que satisfaz o princípio da superposição é chamada uma função linear.A superposição pode ser descrita por dois princípios mais simples: adição
\begin{equation}
    F(x_1 + x_2) = F(x_1) + F(x_2)
\end{equation}
e homogeneidade
\begin{equation}
    F(\alpha x) = \alpha F(x)
\end{equation}
para um \(\alpha\) escalar.
Uma das maneiras de calcular computacionalmente o comportamento de uma função de uma onda é escrevê-la como uma superposição conhecida como superposição quântica.Como a equação de Schrödinger é uma equação linear pode ser utilizada para calcular este comportamento.
\begin{equation}
    |\psi_i\rangle = \sum_{j}C_j|\phi_j\rangle
\end{equation}
onde \(C_j \epsilon C\) e \(|\phi_j\rangle\) pertence a uma base ortonormada.Como tal um objeto pode estar em dois estados ao mesmo tempo enquanto permanece um único objeto o que possibilita qubits a estarem em dois estados:
\newline\begin{equation}
    |0\rangle:=\begin{pmatrix}1\\0\end{pmatrix}
\end{equation}
\newline Estado zero
\newline\begin{equation}
    |1\rangle := \begin{pmatrix}0\\1\end{pmatrix}
\end{equation}
\newline Estado um
\newline\newline Um qubit pode então ser encontrado numa qualquer superposição quântica dos dois estados clássicos 
\(|0\rangle e |1\rangle:\)
\newline
\begin{equation}
        |\psi\rangle := \alpha |0\rangle + \beta |1\rangle =\begin{pmatrix}
    \alpha \\ \beta 
\end{pmatrix};|\alpha|^2+|\beta|^2 = 1
\end{equation}
\section{Mecânica quântica}
A mecânica quântica é uma teoria fundamental da física que fornece uma descrição das propriedades físicas da natureza na escala de átomos e partículas subatômicas.
\newline
A mecânica quântica difere da física clássica porque a energia , o momento , o momento angular e outras quantidades de um sistema limitado são restritas a valores discretos ( quantização ); os objetos têm características de partículas e ondas ( dualidade onda-partícula ); e há limites para a precisão com que o valor de uma quantidade física pode ser previsto antes de sua medição, dado um conjunto completo de condições iniciais (o princípio da incerteza ).
A mecânica quântica surgiu gradualmente de teorias para explicar observações que não podiam ser conciliadas com a física clássica, como a solução de Max Planck em 1900 para o problema da radiação do corpo negro e a correspondência entre energia e frequência no artigo de Albert Einstein de 1905 que explicou o efeito fotoelétrico . Essas primeiras tentativas de entender os fenômenos microscópicos, agora conhecidos como a " velha teoria quântica ", levaram ao pleno desenvolvimento da mecânica quântica em meados da década de 1920 por Niels Bohr , Erwin Schrödinger , Werner Heisenberg , Max Born ,Paulo Dirac e outros. A teoria moderna é formulada em vários formalismos matemáticos especialmente desenvolvidos . Em um deles, uma entidade matemática chamada função de onda fornece informações, na forma de amplitudes de probabilidade ,sobre quais medições da energia, momento e outras propriedades físicas de uma partícula podem resultar.
\subsection{Quantização}
Na física , a quantização (em inglês britânico quantização ) é o procedimento sistemático de transição de uma compreensão clássica dos fenômenos físicos para uma compreensão mais recente conhecida como mecânica quântica . É um procedimento para construir a mecânica quântica a partir da mecânica clássica .
\subsection{Dualidade onda-partícula}
A dualidade onda-partícula é o conceito na mecânica quântica de que cada partícula ou entidade quântica pode ser descrita como uma partícula ou uma onda.Através do trabalho de Max Planck , Albert Einstein , Louis de Broglie , Arthur Compton , Niels Bohr , Erwin Schrödinger e muitos outros, a teoria científica atual sustenta que todas as partículas exibem uma natureza ondulatória e vice-versa.Este fenômeno foi verificado não apenas para partículas elementares, mas também para partículas compostas como átomos e até moléculas. 
\subsection{Princípio da incerteza}
Na mecânica quântica,o princípio da incerteza (também conhecido como princípio da incerteza de Heisenberg ) é qualquer uma de uma variedade de desigualdades matemáticas afirmando um limite fundamental para a precisão com que os valores de certos pares de quantidades físicas de uma partícula,como a posição,x,e momento,p, podem ser previstos a partir das condições iniciais.Introduzido pela primeira vez em 1927 pelo físico alemão Werner Heisenberg,o princípio da incerteza afirma que quanto mais precisamente a posição de alguma partícula é determinada, menos precisamente seu momento pode ser previsto a partir das condições iniciais e vice-versa. No artigo publicado em 1927,Heisenberg originalmente concluiu que o princípio da incerteza era\(\Delta p \Delta q\)usando a constante de Planck completa.A desigualdade formal relacionando o desvio padrão da posição \(\sigma_x\) e o desvio padrão do momento \(\sigma_p\) foi derivada por Earle Hesse Kennard mais tarde naquele ano e por Hermann Weyl em 1928:
\begin{equation}
    \sigma_x \sigma_p \geq \frac{\hbar}{2}
\end{equation}
\section{Entanglement quântico}
Entanglement quântico é o fenômeno que ocorre quando um grupo de partículas é gerado, interage ou compartilha proximidade espacial de tal forma que o estado quântico de cada partícula do grupo não pode ser descrito independentemente do estado das demais,inclusive quando as partículas estão separados por uma grande distância.
\newline
Um dos princípios da computação quântica diz que dois sistemas que estão muito distantes para influenciar um ao outro podem,no entanto,se comportar de maneiras que,embora individualmente aleatórias,são de alguma forma fortemente correlacionadas .
A ideia central por trás do segundo princípio é o entanglement quântico. Ao ler o princípio, alguém pode estar inclinado a pensar que o entanglement é simplesmente uma forte correlação entre duas entidades,mas o entanglement vai muito além da mera correlação perfeita (clássica). Se duas pessoas lerem o mesmo jornal, terão aprendido a mesma informação.Se uma terceira pessoa vier e ler o mesmo jornal, ela também terá aprendido essa informação.Todas as três pessoas neste caso estão perfeitamente correlacionadas e permanecerão correlacionadas mesmo se estiverem separadas umas das outras.
O entanglement quântico é um pouco mais subtil.No mundo quântico,duas pessoas poderíamos ler o mesmo papel quântico e,no entanto, não saberiam que informação está realmente contida no papel até que se reúnam e compartilhem as informações.No entanto, quando estam juntos, descobrem que se pode extrair mais informações do papel do que se pensava ser possível inicialmente. Assim, o entanglement quântico vai muito além da correlação perfeita.
\newline
Um entangled system é definido como aquele cujo estado quântico não pode ser fatorado como um produto de estados de seus constituintes locais; isto é, não são partículas individuais, mas um todo inseparável. No entanglement, um constituinte não pode ser totalmente descrito sem considerar o(s) outro(s). O estado de um sistema composto é sempre expresso como uma soma, ou superposição , de produtos de estados de constituintes locais; ele é entangled se essa soma não puder ser escrita como um único termo de produto.
\newline
Como um exemplo de entanglement: uma partícula subatômica decai em um par entangled de outras partículas. Os eventos de decaimento obedecem a várias leis de conservação e, como resultado, os resultados de medição de uma partícula filha devem ser altamente correlacionados com os resultados de medição da outra partícula filha (de modo que os momentos totais, momentos angulares, energia e assim por diante permaneçam aproximadamente o mesmo antes e depois deste processo). Por exemplo, uma partícula de spin zero pode decair em um par de partículas de spin 1/2. Uma vez que o spin total antes e depois desse decaimento deve ser zero (conservação do momento angular), sempre que a primeira partícula é medida como spin up em algum eixo, o outro, quando medido no mesmo eixo, sempre encontra-se com rotação para baixo . (Isso é chamado de caso anticorrelacionado de spin; e se as probabilidades anteriores para medir cada spin forem iguais, diz-se que o par está no estado singleto .)
\newline
O resultado acima pode ou não ser percebido como surpreendente.Um sistema clássico exibiria a mesma propriedade, e uma teoria de variável oculta certamente seria necessário fazê-lo, com base na conservação do momento angular tanto na mecânica clássica quanto na mecânica quântica. A diferença é que um sistema clássico tem valores definidos para todos os observáveis o tempo todo, enquanto o sistema quântico não. Em um sentido a ser discutido abaixo, o sistema quântico considerado aqui parece adquirir uma distribuição de probabilidade para o resultado de uma medição do spin ao longo de qualquer eixo da outra partícula após a medição da primeira partícula. Esta distribuição de probabilidade é em geral diferente do que seria sem a medição da primeira partícula.


\chapter{Aplicações da computação quântica}
\label{chap.aplicacoes}
Neste capitulos são apresentadas várias aplicações da computação quântica, sendo o desenvolvimento de novos medicamentos e pesquisa de novos tratamentos uma destas aplicações. Cada aplicação será apresentada num subcapitulo próprio.
Com o crescimento exponencial em poder de computaçao a realidade dos computadores quânticos está cada vez mais próxima. Os computadores quânticos  são ideais para resolver problemas complexos, difíceis para computadores clássicos, mas fáceis de fatorar em um computador quântico. Tal avanço cria um mundo de oportunidades, em quase todos os aspectos da vida moderna.
Recentemente a Google proclamou supremacia quântica ao executar um teste com o seu computador quântico Sycamore, teste onde ficou provado que o computador quântico da empresa americana conseguiu executar em poucos segundos uma tarefa que demoraria diversos anos para um computador clássico.
\section{Aplicaçôes na cibersegurança}
\begin{enumerate}
    \item A geração de números aleatórios quânticos é fundamental para a criptografia. Os geradores de números aleatórios convencionais geralmente dependem de algoritmos conhecidos como geradores de números pseudo-aleatórios, que não são verdadeiramente aleatórios e, portanto, potencialmente abertos a comprometimento. Empresas como Quantum Dice e IDQuantique estão desenvolvendo geradores quânticos de números aleatórios que utilizam óptica quântica para gerar fontes de verdadeira aleatoriedade. Esses produtos já estão sendo implantados comercialmente.
    \item A aplicação mais controversa da computação quântica é seu potencial para quebrar a criptografia de chave pública, especificamente o algoritmo RSA.
    O RSA baseia se no facto de que o produto de dois números primos é computacionalmente difícil de fatorar. Um computador clássico levaria triliões de anos para quebrar a criptografia RSA. Um computador quântico com cerca de 4.000 qubits sem erros poderia derrotar o RSA em segundos. No entanto, isso exigiria cerca de 1 milhão dos qubits de hoje. O maior computador quântico do mundo tem atualmente menos de 100 qubits ; no entanto, a IBM e o Google têm planos para atingir 1 milhão até 2030.
\end{enumerate}
A computação quântica promete transformar a segurança cibernética, mas há desafios substanciais a serem enfrentados e avanços fundamentais ainda precisam ser feitos. O desafio mais imediato é alcançar um número suficiente de qubits tolerantes a falhas para liberar a promessa computacional da computação quântica.
\section{Aplicações médicas}
\begin{enumerate}
    \item Como referi antes a computação quântica vem com várias vantagens para a cibercegurança nomeadamente para e encriptaçãoos dados de um paciente são importantíssimos e devem ser guardados para que possa ser efetuado um melhor acompanhamento e tratamento do paciente e como tais dados são confidenciais seriam guardados utilizando encriptação quântica.
    \item Devido á velocidade de execução de algoritmos destes computadores seria muito mais rápido executar análises e com muita mais precisão o que levaria a melhores diagnósticos e por consequência mais vidas serâo salvas.
    \item Esta tecnologia pode também ser aplicada a radioterapia no sentido em que o computador quântico seria capaz de executar os tratamentos com mais precisão.
\end{enumerate}
Computadores quânticos prometem melhorar imenso análise e encriptação de dados e aumentar bastante as capacidades de simulação, a eficiência e a precisão de tratamentos atuais.
\section{Aplicações farmacêuticas}
\begin{enumerate}
    \item Esta tecnologia permite modelar as interações droga-alvo necessárias. Esta inovação ajuda a comparar como moléculas complexas podem interagir umas com as outras em um futuro medicamento. Além disso, faz cálculos mesmo com compostos cuja fórmula completa é desconhecida. Isso ajuda a prever reações positivas e negativas em uma situação de alta incerteza.
    \item Permite examinar grandes bibliotecas virtuais. Dessa forma, um pesquisador pode selecionar os compostos mais relevantes com muito mais eficiência e rapidez.
    \item Reduz a necessidade de testes in vitro caros. Em vez de trabalhar com amostras como sangue ou tecido humano, uma empresa farmacêutica pode criar pacientes virtuais, ou seja, uma simulação de computador interativa usada para estudar reações a medicamentos.
\end{enumerate}
A descoberta de medicamentos é um processo caro. O desenvolvimento de um único medicamento é estimado em 2 biliôes de euros. Enquanto uma em cada 10 chega ao mercado, as outras tentativas falham em um dos estágios de desenvolvimento clínico. A computação quântica na indústria farmacêutica pode reduzir esses custos e levar a medicamentos mais baratos a longo prazo.



\chapter{Diferenças entre computadores quânticos e clássicos}
\label{chap.diferencas}
Neste capitulo serão abordadas diferenças entre computadores quânticos e computadores clássicos. De modo a deixar o relatorio mais organizado, posteriormente
serão criados subcapítulos, onde cada diferença será apresentada num subcapítulo proprio.
Algo que será abordado será o algoritmo de Shor, um algoritmo quantico que é utilizado para fatorizar um número primo de L bits.

\chapter{Típos de Computadores Quânticos}
\label{chap.tipos}
Computador quântico analógico e digital, computador quântico universal e computador quântico adiabático são os três tipos de computadores quânticos abordados neste capítulo.

\chapter*{Contribuições dos autores}
A escolha de tema, de capítulos e a contrução da introdução foi realizada por ambos os autores.
\paragraph{} Usar abreviaturas para identificar os autores,
JP para João Pinto e GB para Gabriel Boia.

\vspace{10pt}
\textbf{Indicar a percentagem de contribuição de cada autor.}\\

\autores : x\%, x\%\\

%%%%%%%%%%%%%%%%%%%%%%%%%%%%%%%%%
\chapter*{Acrónimos}
\begin{acronym}
\acro{jp}[JP]{João Pinto}
\acro{gb}[GB]{Gabriel Boia}
\acro{cq}{CQ}{Computadores Quânticos}
\acro{ua}[UA]{Universidade de Aveiro}
\acro{leci}[LECI]{Licenciatura em Engenharia de Computadores e Informática}
\end{acronym}


%%%%%%%%%%%%%%%%%%%%%%%%%%%%%%%%%
\printbibliography

\end{document}
